\chapter{Confronto esplicito}
\label{chap:confronto_esplicito}

Alla luce delle considerazioni evidenziate nei capitoli \ref{chap:background} e \ref{chap:protocolli_qkd} e del funzionamento dei protocolli, possiamo confrontarli esplicitamente sulla base delle dimensioni di confronto individuate e descritte nel capitolo \ref{chap:metodologie}.
\begin{table}[h!]
    \centering
    \begin{tabular}{||l | l l l l||} 
        \hline
         & \textbf{BB84} & \textbf{E91} & \textbf{B92} & \textbf{SARG04} \\ \hline
        \textbf{Principio quantistico} & Heisenberg & Entanglement & Heisenberg & Heisenberg \\ 
        \textbf{Numero di stati} & 4 & 2 coppie EPR & 2 & 4 \\ 
        \textbf{Polarizzazione} & ortogonale & ortogonale & non ortogonale & ortogonale \\ 
        \textbf{DoS} & vulnerabile & vulnerabile & vulnerabile & vulnerabile \\
        \textbf{PNS} & vulnerabile & vulnerabile & vulnerabile & meglio di BB84 \\
        \textbf{Intercept/resend} & vulnerabile & vulnerabile & vulnerabile & vulnerabile \\
        \textbf{Beam splitting} & vulnerabile & vulnerabile & vulnerabile & vulnerabile \\
        \textbf{Side channel} & vulnerabile & vulnerabile & vulnerabile & vulnerabile \\
        \hline
    \end{tabular}
    \caption{Confronto dei protocolli di QKD.}
    \label{table:confronto_esplicito}
\end{table}
\\ Dalla tabella \ref{table:confronto_esplicito} possiamo constatare che:
\begin{itemize}
    \item la prevalenza dei principali algoritmi si basa sul principio di indeterminazione. Solo E91 si fonda sull'Entanglement;
    \item il numero di stati oscilla tra 2 e 4;
    \item la polarizzazione dei fotoni maggiormente adottata fa uso di basi i cui stati sono ortogonali;
    \item sebbene tutti i protocolli siano idealmente incondizionatamente sicuri \cite{mayers}, \cite{tamaki} \cite{lo_chau}, sono tutti vulnerabili agli attacchi descritti in precedenza (ad eccezione di. Questo implica che nella pratica la sicurezza incondizionata non sempre è facile da raggiungere \cite{brassard_lutknenhaus}. Non sono stati ancora trovati meccanismi di difesa validi e, come già indicato, per alcuni attacchi non è possibile implementare una controffensiva definitiva, ossia che garantisca la sicurezza nei confronti di essi.  
\end{itemize}

SARG04 appare attualmente come il miglior protocollo tra quelli indicati perché, nonostante le vulnerabilità, fornisce una robustezza maggiore dei concorrenti. Ad esempio, SARG04 ha una resistenza maggiore all'attacco PNS, per via del fatto che viene implementato utilizzando laser ad impulsi invece che sorgenti a singoli fotoni. Sebbene l'impiego degli impulsi non lo rendano automaticamente immune agli attacchi di tipo \textit{beam splitting}, abbiamo osservato nel capitolo \ref{chap:background} alcuni meccanismi di difesa. Un ulteriore vantaggio dal punto della sicurezza si può osservare a livello algoritmico in quanto Alice non rivela direttamente le proprie basi a Bob su un canale classico, bensì rivela una coppia di stati non ortogonali in cui i bit vengono codificati. 

Dal momento che tutti i protocolli sono vulnerabili ad attacchi di tipo \textit{intercept-resend}, emerge come la probabilità con la quale Eve può indovinare la corretta base di polarizzazione per misurare i qubit che intercetta debba essere piccola. Per garantire probabilità ridotte a tale scopo, è necessario aumentare il numero di basi di polarizzazione. Infatti, se venissero utilizzate tre basi di coppie di stati quantici ortogonali per la codifica dei bit, la probabilità con cui Eve potrebbe indovinare la corretta base da utilizzare per la misurazione dei qubit intercettati sarebbe del 33\% (contro quella del 50\% nei casi in cui si utilizzino solamente due basi). A tal proposito, ricordiamo che nel capitolo precedente sono state accennate delle versioni di E91 con 3 basi di polarizzazione. Dunque, un fattore fondamentale della sicurezza è preferire quei protocolli di QKD (o varianti degli stessi) con un numero di stati superiore. 

Ciò nonostante, il problema principale che accomuna tutti i protocolli di QKD visti è la verifica dell'identità delle due parti, l'autenticazione. Infatti, il livello di sicurezza garantito per le comunicazioni entro un protocollo di QKD non si cura della vera identità della parti coinvolte: uno tra Alice e Bob potrebbe essere sostituito da Eve facendo credere all'altro di comunicare con una entità differente. L'autenticazione può essere ottenuta sovrapponendo al protocollo di QKD un secondo protocollo oppure per mezzo di terze parti, come il TC.

Inseriamo un'analisi aggiuntiva rispetto alle dimensioni di confronto appena viste, che tiene conto delle effettive applicazioni dei protocolli di QKD.
\begin{table}[h!]
    \centering
    \begin{tabular}{||l | l l l l||} 
        \hline
         & \textbf{BB84} & \textbf{E91} & \textbf{B92} & \textbf{SARG04} \\ \hline
        \textbf{Implementazioni simulate} & sì & sì & sì & sì \\ 
        \textbf{Implementazioni reali} & sì & sì & sì & sì \\ 
        \textbf{Applicazioni industriali} & n/d & n/d & n/d & n/d \\ 
        \hline
    \end{tabular}
    \caption{Implementazioni e sbocchi applicativi dei protocolli di QKD.}
    \label{table:implementazioni}
\end{table}
\\ La tabella \ref{table:implementazioni} sottolinea che esistono versioni simulate dei suddetti protocolli: queste principalmente le ritroviamo sviluppate con toolbox di Matlab \cite{matlab_impl}, oppure con i servizi offerti da IBM Quantum. Esistono anche implementazioni reali dei protocolli trattati che sfruttano sorgenti di singoli fotoni o laser ad impulsi e dove il mezzo trasmissivo è la fibra ottica, conduttori o lo spazio vuoto \cite{knight} \cite{hughes} \cite{ent_impl}. Ritroviamo infine realtà aziendali come  IDQuantique, QuintessenceLabs, MagicQ e SeQureNet che dichiarano di fornire soluzioni hardware (ad esempio generatori quantici di numeri casuali) e software in ambito di quantum computing, senza dichiarare tuttavia i protocolli utilizzati.

A livello sperimentale il BB84 appare come il protocollo più quotato per effettuare test di funzionamento su territorio nazionale. Sulla spinta della Commissione Europea, il Joint Research Center in \cite{in_field_implementations} ha redatto un resoconto dello stato dell'arte delle sperimentazioni di ciascuno Stato. È emerso che BB84 è stata la scelta primaria essendo uno dei primi protocolli ad essersi imposto in ambito QKD, dunque uno dei più diffusi. 

Riguardo ad aspetti tecnico-pratici, i protocolli che si basano sulla fibra ottica sono molto utilizzati. Ad oggi, \cite{fibre421} riporta la lunghezza massima della fibra ottica (421 km) che si può raggiungere prima che i disturbi e le perdite a cui è soggetta non garantiscano la terminazione desiderata dei protocolli di QKD (posto che la fibra ottica sia ad alte prestazioni). Emerge quindi nuovamente come il canale trasmissivo sia un punto critico delle implementazioni dei protocolli di QKD. Per quanto riguarda invece la scelta di impiego di sorgenti a singoli fotoni rispetto a laser ad impulsi, la scelta più saggia dovrebbe ricadere sugli ultimi per via della maggiore robustezza verso gli attacchi di tipo \textit{beam splitting} e \textit{Denial of Service}, dal momento che è più facile prevenire attacchi DoS quando si usano gruppi di fotoni (impulsi) che non singoli fotoni per la codifica dell'informazione (a meno di danni fisici ai canali di trasmissione).\\
Altre implementazioni, come accennato, sfruttano l'aria o il vuoto come mezzo di trasmissione \cite{knight} ma le applicazioni rimangono limitate.\\
Diversamente, invece, le implementazioni dei protocolli basati su QE consentono distanze maggiori fra le parti della comunicazione perché sfruttano il principio di \textit{action at distance} che caratterizza il comportamento di particelle "entangled" \cite{ent_impl}.