\begin{document}

\section{Scopo del lavoro}
%What do you want to research?
Lo scopo di questa trattazione è l'investigazione di protocolli e sistemi quantistici capaci di contrastare le azioni di un attaccante che dispone di tecnologie quantistiche arbitrarie. All'interno di questo documento verrà descritto il funzionamento di protocolli quantistici di scambio di chiavi crittografiche, relativi punti di forza e debolezza, vulnerabilità e meccanismi di difesa per contrastare lo sfruttamento delle stesse da parte degli attaccanti. Verranno inoltre illustrati, e approfonditi in modo analogo a quanto sopra descritto, protocolli quantistici di \textit{Blind Quantum Computing} per lo sfruttamento di risorse computazionali quantistiche remote.\\

%Why that topic? It must consider:
% - originality: must have some elements of novelty
% - impact: must benefit some stakeholders. If so, specifify who is your intended audience
%Edoardo
Lo scopo di questa trattazione è la promozione e la valorizzazione delle tecniche di sicurezza quantistiche che, se impiegate, portano alla risoluzione di numerosi problemi che affliggono, o affliggeranno in futuro, gli odierni sistemi impiegati nella sicurezza informatica delle comunicazioni e delle computazioni delegate a macchine remote. Il documento intende riportare in modo conciso e fruibile una revisione completa e critica di quanto sopra menzionato. \\
Il pubblico a cui è rivolto comprende gli utilizzatori degli algoritmi tradizionali di crittografia interessati ai miglioramenti della disciplina per mezzo della \textit{quantum criptography}, nonché quanti sono interessati a sfruttare la potenza computazionale quantistica su cloud in modo sicuro, riservato e verificabile.\\

%Rifio & Alberto
%Il gruppo di stakeholders che, in primis, condivide l'interesse per la suddetta materia è l'insieme di individui che usufruiscono di algoritmi tradizionali (non quantistici) vulnerabili nei confronti di attacchi quantistici. Inoltre è utile a coloro che desiderano sfruttare il cloud computing quantistico in quanto possono garantire la confidenzialità delle elaborazioni dei propri clienti tramite il Blind Computing quantistico. \\
%La cifratura è un meccanismo universalmente utilizzato, quindi riteniamo opportuno non delineare categorie di utenti specifiche.

%How do you want to research? This should be detailed in comparison with the rest of the literature
%Edoardo
Per la stesura della trattazione verranno considerate in primo luogo le revisioni generali recenti che offrono una vasta panoramica dello stato dell'arte degli argomenti trattati. In secondo luogo, verranno analizzati gli articoli scientifici citati all'interno di queste ultime. Infine, si estenderà la ricerca di fonti autorevoli utili agli scopi della trattazione a noti motori di ricerca quali Elsevier, Research Gate, IEEE Xplore e Google Schoolar.\\
Di seguito elenchiamo una prima lista non definitiva degli articoli scientifici considerati:
 %qui metterei le review RECENTI e qualche paper specifico di qualche protocollo


% TODO: mettere bibliografia
% "This should be detailed in comparison with the rest of the literature" per Alby e Rifio significa "Aggiungete le fonti"
\end{document}